\chapter{绪论}

\section{研究背景与意义}

近年来,我国高度重视人工智能与智能感知技术的发展,将其作为推动经济社会高质量发展的重要引擎。
2025年,国务院印发《关于深入实施“人工智能+”行动的意见》,明确提出要加快人工智能与实体经济、社会治理和公共服务的深度融合,
重点推进智能感知、智能决策与智能控制等关键技术在交通运输、公共安全、智慧城市、无人系统等典型场景中的规模化应用\cite{AI_Plus_Action_2025}。
在此国家战略背景下,如何提升人工智能系统在真实复杂环境中的感知能力与运行可靠性,已成为当前学术界和产业界共同关注的核心问题。

多目标检测与跟踪作为计算机视觉与智能感知领域的重要研究方向,是实现环境理解、行为分析和智能决策的基础技术之一。
该技术通过对视频序列中多个目标进行持续识别、定位与身份保持,为上层决策系统提供结构化的时空信息支撑,已在多个实际场景中得到广泛应用。
\autoref{fig:ch1_1}给出了多目标检测与跟踪技术在典型应用场景中的示例,包括智能交通、自动驾驶、体育赛事分析以及无人系统等方向。
% 随着智慧交通、低空经济和无人系统的快速发展,这些应用场景对多目标检测与跟踪算法在复杂环境下的稳定性、鲁棒性与泛化能力提出了更高要求,其性能直接关系到智能系统在真实场景中的运行安全与决策可靠性。

\begin{figure}[htbp]
    \centering
    \includegraphics[width=12cm]{chapter1/1.png}
    \caption{\label{fig:ch1_1}多目标跟踪应用实例}
\end{figure}

1. 智能交通。根据交通运输部的统计数据,2024年我国完成城市客运量达到1067.97亿人次,较2023年增长5.7\%\cite{MOT_2024_Stats}。
面对如此庞大的人流与车流规模,传统依赖人工或规则的交通管理方式已难以满足高效、精细化管理需求。
基于多目标检测与跟踪技术的智能交通系统通过对道路交通参与者进行实时感知与持续跟踪,可为交通流量分析、拥堵预警和交通信号控制优化提供关键数据支撑,
已成为了现代交通管理的重要技术手段。如百度公司与中山大学合作开发了一种在线运动车辆技术系统\cite{Lu2021VehicleCounting},该系统旨在对交通复杂、拥堵的十字路口进行车流量分析。
通过实时监控和数据分析,该系统能够提供准确的车流信息,有助于交通管理部门在高峰期进行科学的交通调控,缓解路口拥堵,提升整体交通流效率。
同时,这一系统的引入也有助于提高路口的车辆通行能力和信号灯的工作效率,为进一步优化城市交通管理提供了技术支持。

2. 自动驾驶。车辆需要在动态、复杂且高度不确定的交通环境中实现安全行驶,这对环境感知系统提出了极高要求。
多目标跟踪技术通过持续感知并预测周围行人、车辆等多类目标的运动状态,为自动驾驶系统的路径规划和决策控制提供基础信息,是实现高等级自动驾驶不可或缺的关键技术之一。
如广汽研究院X lab团队提出了一种创新的多目标跟踪算法,在自动驾驶行业内首次将跟踪的多视角数据通过深度学习方法转换到鸟瞰图特征空间下,并在国际权威的自动驾驶测试竞赛中获得纯视觉榜单全球第一。

3. 体育应用。体育赛事因其精彩纷呈而广受大众喜爱,一般观众通常关注精彩瞬间或喜爱的运动员,而专业球迷、运动员、教练以及媒体从业者则更倾向于比赛的深入分析。
为满足这些不同需求,多目标跟踪技术在体育比赛分析中发挥了重要作用。通过对球员和球体的精准跟踪,该技术为运动员表现评估、战术调整、团队合作优化以及比赛再编辑提供了科学依据,
提升了比赛的观赏性和技术分析的精准性。如2022年的北京冬奥会引入的“3D运动员跟踪技术”,该技术基于多目标跟踪算法,能够实时检测并捕捉运动员的速度、加速度、运动方向和运动轨迹等数据,
进而生成运动员的3D形态信息。通过这些数据的分析,该技术能够提取并生成每个运动员的精彩比赛瞬间,并在直播过程中即时回放,为观众带来更加沉浸式的观看体验。

4. 无人机。随着无人机技术的不断成熟,其在军事和民用领域的价值不断凸显。在军事方面,无人机不仅提升了战场感知能力,其实时性强、隐蔽性好、抗干扰能力佳的特点,使它能够快速、准确地收集情报,
同时避免人员直接暴露于危险之中;在影视制作和体育赛事中,无人机能够自动跟随拍摄对象,提供稳定、流畅的空中拍摄画面;在搜索救援中,无人机可以自动跟随搜救人员,提供空中视角,
帮助搜救人员更高效地开展救援工作。目前,目标跟踪技术已在商业领域实现实际应用,并具备实时对拍摄对象进行自动对焦的能力。以大疆创新公司为例,其开发的无人机产品如Mini 3、Air 3等,
均配备了先进的智能跟随系统,允许无人机能够在无需人工干预的情况下自动跟随特定目标。

尽管多目标跟踪技术在上述场景中展现出巨大潜力,但在实际部署过程中仍面临诸多挑战。
真实环境中普遍存在光照变化剧烈、天气条件复杂、目标密集遮挡以及相机运动等问题,使得目标外观特征和运动模式呈现出显著的不确定性,
导致现有算法在跨环境条件下容易出现目标身份切换频繁、跟踪中断等现象,严重影响系统的稳定性与可靠性。
因此,如何在复杂多变的环境条件下有效建模目标之间的时空关系,实现鲁棒、高效且具备良好泛化能力的多目标跟踪,已成为当前亟待解决的关键科学与工程问题。


\section{国内外研究现状}


\subsection{基于深度学习的多目标跟踪算法概述}

\subsection{图神经网络在多目标跟踪中的应用}

\subsection{跨域视觉感知与自适应方法研究}

\section{本文主要创新点与贡献}


\section{论文结构安排}
