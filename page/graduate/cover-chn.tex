\thispagestyle{cover}

\begin{center}
    \zihao{-4} \songti
    \begin{tabularx}{\textwidth}{l l >{\raggedleft}X l}
        分类号: &
        \underline{\makebox[2cm][c]{TP391.4}}  &
        单位代码: &
        \underline{\makebox[2cm][c]{10335}} \\

        密{\quad}级: &
        \underline{\makebox[2cm][c]{}} &
        学{\quad\quad}号: &
        \underline{\makebox[2cm][c]{22332090}}
    \end{tabularx}
\end{center}



\begin{center}
    \includegraphics[width=0.5\paperwidth]{logo/zjuchar.pdf}
\end{center}

\vspace{-40pt}

\begin{center}
    \zihao{-1} \songti%
    \TitleTypeNameCover
\end{center}

\vskip 15pt

\begin{center}
    \includegraphics[width=0.15\paperwidth]{logo/zju.pdf}
\end{center}

\vskip 15pt

% Define multi line title

\ifnumcomp{\TitleLines+\TitleEngLines}{=}{5}{
    \renewcommand{\arraystretch}{0.7}
}{
    \ifnumcomp{\TitleLines+\TitleEngLines}{=}{6}{
        \renewcommand{\arraystretch}{0.65}
    }
}

\begin{center}
    \bfseries \zihao{-2}
    % 修改点1:将宽度从 .8\textwidth 改为 1.0\textwidth,防止小二号字太宽导致溢出
    \begin{tabularx}{1.0\textwidth}{>{\fangsong}l X<{\centering}} 
        
        % --- 中文题目部分 ---
        \ifthenelse{\equal{\TitleLines}{1}}
        {
            % TitleLines == 1
            中文论文题目:&  \uline{\hfill \fangsong \Title{} \hfill} \\
            ~ & \uline{\hfill} \\
        }
        {
            \ifthenelse{\equal{\TitleLines}{2}}
            {
                % TitleLines == 2
                中文论文题目:&  \uline{\hfill \fangsong \TitleLineOne{} \hfill} \\
                ~            & \uline{\hfill \fangsong \TitleLineTwo{} \hfill} \\
            }
            {
                % TitleLines == 3
                中文论文题目:&  \uline{\hfill \fangsong \TitleLineOne{} \hfill} \\
                ~            & \uline{\hfill \fangsong \TitleLineTwo{} \hfill} \\
                ~            & \uline{\hfill \fangsong \TitleLineThree{} \hfill} \\
            }
        }
        
        % 修改点2:使用 \noalign{\vspace{...}} 强力调整行间距
        % 这里设置为 -1.5cm,你可以尝试 -1cm 或 -2cm 直到满意为止
        % 这种写法比 [-1em] 更稳定,不受 ifthenelse 换行空格的影响
        \noalign{\vspace{0.3cm}}
        
        % --- 英文题目部分 ---
        \ifthenelse{\equal{\TitleEngLines}{1}}
        {
            % TitleEngLines == 1
            英文论文题目:& \zihao{3} \uline{\hfill \TitleEng{} \hfill} \\
            ~ & \uline{\hfill} \\
        }
        {
            \ifthenelse{\equal{\TitleEngLines}{2}}
            {
                % TitleEngLines == 2
                英文论文题目:& \zihao{3} \uline{\hfill \TitleEngLineOne{} \hfill} \\
                ~            & \zihao{3} \uline{\hfill \TitleEngLineTwo{} \hfill} \\
            }
            {
                % TitleEngLines == 3
                英文论文题目:& \zihao{3} \uline{\hfill \TitleEngLineOne{} \hfill} \\
                ~            & \zihao{3} \uline{\hfill \TitleEngLineTwo{} \hfill} \\
                ~            & \zihao{3} \uline{\hfill \TitleEngLineThree{} \hfill} \\
            }
        }
    \end{tabularx}
\end{center}


\vskip 30pt


\begin{center}
    \zihao{4}
    \begin{tabularx}{.6\textwidth}{>{\fangsong}l >{\fangsong}X<{\centering}}
        \ifthenelse{\equal{\BlindReview}{true}}%
        {%
            申请人姓名: & \uline{\hfill} \\
            指导教师:   & \uline{\hfill} \\
            合作导师:   &  \uline{\hfill} \\
        }
        {%
            申请人姓名: & \uline{\hfill \StudentName \hfill} \\
            指导教师:   & \uline{\hfill \AdvisorName \hfill} \\
            合作导师:   &  \uline{\hfill \ColaboratorName \hfill} \\
        }
        \ifthenelse{\equal{\Type}{design}}
        {%
            专业学位类别:  &  \uline{\hfill \Major \hfill} \\
            专业学位领域:  &  \uline{\hfill \Topic \hfill} \\
        }
        {%
            学科(专业):  &  \uline{\hfill \Major \hfill} \\
            研究方向:  &  \uline{\hfill \Topic \hfill} \\
        }
        \ifthenelse{\equal{\DepartmentLines}{1}}
        {%   DepartmentLines == 1
        所在学院:   &  \uline{\hfill \Department \hfill} \\
        }
        {%   DepartmentLines == 2
        所在学院:   &  \uline{\hfill \DepartmentLineOne \hfill} \\
                    &  \uline{\hfill \DepartmentLineTwo \hfill} \\
        }
    \end{tabularx}
\end{center}

\ifthenelse{\equal{\DepartmentLines}{1}}
{
    %   DepartmentLines == 1
    \vskip 20pt
}
{
    %   DepartmentLines == 2
    \vskip 10pt
}

\begin{center}
    \zihao{-3} \bfseries
    \begin{tabularx}{.6\textwidth}{>{\fangsong}l >{\fangsong}X<{\centering}}
        论文递交日期 & \uline{\SubmitDate}
    \end{tabularx}
\end{center}
